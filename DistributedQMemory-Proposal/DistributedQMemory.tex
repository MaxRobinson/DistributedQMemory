\documentclass[jair,twoside,11pt,theapa]{article}
\usepackage{jair, theapa, rawfonts}

%\jairheading{1}{2018}{}{}{}
\ShortHeadings{Distributed Q-Learning (DQL)}
{Robinson}
\firstpageno{1}

\begin{document}

\title{Distributed Q-Learning: A Democratic Learning Process (Proposal)}

\author{\name Max Robinson \email max.robinson@jhu.edu \\
       \addr Johns Hopkins University,\\
       Baltimore, MD 21218 USA
   }

% For research notes, remove the comment character in the line below.
% \researchnote

\maketitle

\begin{abstract}
This proposal suggests studying a new distributed method for reinforcement learning using Q-learning. Reinforcement learning problems generally take a large amount of data and time to reach optimal performance. Early iterations of Q-learning used a single agent for the entire learning process. Since then, asynchronous and distributed reinforcement learning algorithms and frame works have been suggested. The proposed method to study is a distributed Q-learning implementation (DQL). It uses multiple agent-environment pairs with local parameters to learn independently, then update a central Q-Memory. The hypothesis is that the distributed nature of the algorithm will lead to an increased learning rate and a decreased training time to achieve equal or better performance than a standard single Q-learning agent. To test this hypothesis, it is suggested that the algorithm be tested on three different environments in the OpenAI gym library, and the learning rates for differing numbers of parallel agents recorded and compared to the single agent model. 
\end{abstract}

\section{Introduction}
\label{Introduction}
Reinforcement learning can be described simply as the learning process of an agent in an environment trying to reach a goal. 
The agent learns by attempting to maximize a reward. In some situations the learner has very little prior knowledge of the environment. 
The act of maximizing the reward is the learning process.

Reinforcement learning typically requires an agent to follow a $state \to action \to state'$ loop. As a result, the learning process is inherently sequential. 
For large state spaces or complex environments, this process can be slow. A single agent must often experience many iterations of maximizing 
a reward in order to learn in even a simple environment.

A famous example of Reinforcement Learning is Tesauro's TD-Gammon agent.
The best performing agent required 1,500,000 training games to beat one of the best Backgammon players at the time \cite{Tesauro:1995:TDL:203330.203343}.
As a more modern example, (Mnih et al.) developed an algorithm to play Atari 2600 video games called DQN (Mnih et al. 2015). To learn to play each game at a human level or higher,
50 million frames were required to train an agent for each game. Total training on all 49 games in the study took about 38 days of game experience. 

The constraint of a lone agent acting sequentially can create situations where training an agent to learn a task can take an exorbitant amount of time. 
To combat this, researchers have focused on ways to adapt these reinforcement learning algorithm to run in parallel to decrease the amount of time it takes a single agent to learn. 

Recent research has studied speeding up Deep Neural Networks for RL such as DQN (Mnih et al., 2013) and others. Quite a few papers have suggested ways of parallelizing both the computation for these methods as well as distributing actors and learners to run in parallel, which send back gradients to be used to update a centralized parameter store. 

This proposal suggests a step back to explore a slightly more simplistic model of distributed reinforcement learning using Q-learning. It suggests a parallelized and distributed version of the Q-learning algorithm using a traditional Q-Memory, here on in referred to as Distributed Q-learning (DQL). DQL, similar other distributed reinforcement learning approaches, uses multiple separate agent and environments with local copies of parameters and memory to learn, and has a centralized main Q-Memory. Different from other algorithms, however, DQL sends raw Q-values to the Centralize memory which are then combined with the existing values. The centralized Q-memory then updates values, using a linear combination of Q-values and hyper parameters explained in Section \ref{Approach}. An advantage to this approach is that the traditional Q-learning algorithm is being leveraged. This means that the optimality convergence guarantees of Q-learning, even in an asynchronous settings are likely to still hold true.

It is hypothesized that in using this distributed algorithm, there will be an increase in the learning rate of a learner using the centralized Q-memory when compared to a single Q-learning agent, as the number of parallel agents increases. This increased learning rate will reduce the total number of epochs compared to a single agent to achieve similar or better performance. This will also likely reduce the total wall time it takes to train. 

\section{Previous Work}
\label{Literature Survey}
Q-Learning \cite{watkins} has been a foundational algorithm in reinforcement learning, especially after it was shown to have optimal convergence in the limit \cite{qlearning}. Very soon after this, asynchronous methods of Q-learning were being explored.  

\cite{Tsitsiklis1994} studied if Q-learning would still keep it's optimal convergence property in an asynchronous setting. The results showed that the Q-learning algorithm does continue to keep this convergence guarantee in an asynchronous setting, given that old information is eventually discarded through the update process, along with a few other conditions. 

More recently there has been a flurry of work done in parallelizing reinforcement learning algorithms. \cite{MANNION2015956} used a distributed Q-learning algorithm with multiple agents and environments to learn in, applied to traffic signal control. In their algorithm, at every time step the agents update a global Q Matrix in accordance with the Q-learning update equation. 

In addition to using a tabular implementation of Q-learning, function approximation versions using deep neural networks, such as DQN (Mnih et al., 2013), have also explored asynchronous and distributed architectures for learning. 

The General Reinforcement Learning Architecture (Gorila) by (Nair et al., 2015) uses a general, massively distributed architecture of agent, and environments to learn. Each agent in Gorila has its own environment, a copy of the model which is periodically update, a separate replay memory and a learner that aims to learn using a DQN (Mnih et al., 2015). The learner samples data from the memory replay and computes the gradients for the parameters, which are then sent to a centralized parameter server. The parameter server then updates a central copy of the model based on gradients. 

A3C developed by (Minh et al. 2016) collapses Gorila onto a single machine, using multiple processors to parallelize the agents actions in an environment. A3C similarly computes gradients for each agent's model at each time step, locally, but accumulates these gradients over a series of time steps before updating the central model, every so often. This aims to balance computational efficiency with data efficiency. 


\section{Approach}
\label{Approach}
The focus of this proposed experiment will be on testing the distributed Q-learning algorithm described in Section \ref{algorithm}.
The new algorithm and its variations will be tested according to the experimental approach described in \ref{experiments}

\subsection{Distributed Q-learning Algorithm} 
\label{algorithm}
The proposed algorithm to study is a novel distributed and parallelized version of Q-learning, referred to as DQL. 
DQL uses multiple agents environment pairs to learn rather than a single agent and environment. A central Q-memory $Q_c$ server is also kept and is updated 
as the agents learns and completes epochs. In addition to the central Q-memory, there is also a central learning rate $\alpha_c$ which decays according
to the number of updates $Q_c$ has received. 

More formally DQL has $n$ agents $A_1, ... , A_n$. Each agent also has a copy of the environment in which it acts, $E_1, ..., E_n$. Each agent keeps track 
of a local Q-Memory $Q_n$, in addition to local parameters such as the discount factor, learning rate, and $\epsilon$. Each agent then acts and learns inside side of
its environment according to the $\epsilon$-greedy Q-learning algorithm. 

As the agent learns it will send updates to $Q_c$. In order to send updates, A list of $Q_i(s,a)$s that have been modified since the agent last sent updates is kept by the local agent,
$Set_{q_i} = Set(Q'_i(s_,a), ... )$.
The agent then sends updates to $Q_c$ every $\tau$ epochs. $\tau$ is a hyperparameter that can be adjusted to change how frequently each agent reports updates. 

Updates are then performed by the $Q_c$ server each time an update is retrieved. $Q_c$ will be updated according to equation \ref{eq1},
where $\alpha_{c}$ and $\alpha_i$ are the learning rate for $Q_c$ and $A_i$ respectively. States in the update set 
\begin{equation}
\label{eq1}
Q_c(s,a) = (1-\frac{\alpha_{c}^{2}}{\alpha_i})Q_c(s,a) + \frac{\alpha_{c}^{2}}{\alpha_i} Q_i(s,a)
\end{equation}

After updates to $Q_c$ are complete, there are two variations that will be explored in this experiment for updating the local agent $A_i$. 
The first variation will send back a set of updated values from $Q_c$ for just the set of states sent from $A_i$, $Set_{q_i}$. The agent will 
then replace any local values with those from $Q_c$. This has the affect of reconciling just the states that local agent has updated recently 
with possible updates from other agents. 
The second variation, dubbed DQL-All, overwrites $Q_i$ with the entire $Q_c$ memory. This provides information from other agents to a local agent and homogenizes 
the $Q_i$s of the different agents. In this variation, states that were explored by one agent would then be shared with other agents even if that 
agent has not explored the states itself. 

Learning halts after $Q_c$ has been updated a maximum number of times, or performance ceases to improve. To measure the performance of $Q_c$ an agent 
and environment $A_t, E_t$ are created with $Q_c$ as the instantiated Q-memory. The agent then acts in the environment with no learning occurring. 
The performance metric being used is than captured upon completion and the test agent and environment are destroyed. How often the performance of $Q_c$ 
is assessed is configurable and depends upon the fidelity with which the trainer wishes to assess the learning. 


\subsection{Experimental Approach}
\label{experiments}
DQL will be evaluated by running experiments on 3 different environments provided by the OpenAI Gym \cite{gym}.
The 3 environments used will be Taxi-v2, LunarLander-v2, and BipedalWalker-v2. The Taxi-v2 environment is a discrete 
environment, while the LunarLander-v2, and BipedalWalker-v2 are continuous. For the continuous state space environments, 
the values for the states will be discretized on a per environment basis to ensure a single learner can learn in the specific environment. 
Each environment supplies a different level of problem complexity, and as such aims to demonstrate that any performance gains that could
be observed using DQL are not subject to the complexity of the environment.

Experiments in each environment will be done using DQL with 2, 4, and 8 agents and done with each variant DQL and DQL-All . Each experiment will be run with 10 trials. The final results
being averaged over all trials. Each agent will use an $\epsilon$-greedy behavior policy with $\epsilon$ annealed from 1 to .1 over the learning process. The annealing may vary per environment. 

The learning rate for $Q_c$ will be annealed from 1 to .1 over the number of updates from agents. The learning rate per agent is annealed from 1 to .1 over the learning process, updating the learning rate after each epoch. The update frequency for each agent, $\tau$, is anticipated to be set to 1 for all experiments. If additional time is available, $\tau$ will be varied 
to use the values 1, 5, and 10 to inspect the effect that update frequency has on learning. 

\subsubsection{Evaluation}
The performance of DQL and DQL-All will be evaluated based on their learning rate to achieve better or equal performance to a single agent system using Q-learning. 
The learning rate will be measured in the number of epochs required to attain the performance. 

The performance measures to evaluate an algorithm in their environment is determined by the provided OpenAI environment. Each environment has a score mechanic that is calculated
differently for each one. A learners performance will be judged on the maximization of the environments scoring metric. For example if the goal is to travel a long distance and the score 
is based on how far the agent got, the performance for that agent will be judge on maximizing the score, i.e. distance. 

In addition to traditional performance metrics, the number of newly explored states will be tracked for each agent. As the agent discovers a new state, the state will be tagged with which epoch
the state was discovered in. This will then be used to compare the exploration patterns of the agents throughout the training process. Comparing which states are discovered by which agents at what time
will give insight into if or when the agents started to converge to the same policies. In addition, it will provide data to evaluate if parallel exploration of the state space provides value to learning. 

\subsection{Conclusion}
This proposal suggests studying a new variation of distributed reinforcement learning using Q-learning as its foundation. Using multiple individual agent-environment pairs with local parameters and memory along with centralized Q-Memory asynchronously update, it is hypothesized that DQL will decrease the training time required to meet or out perform a standard Q-learning agent. The algorithm will be tested on three different environments with learning rates captured and compared with a single agent Q-learning algorithm to see if this hypothesis is true.

\vskip 0.2in
\bibliography{DistributedQMemory}
\bibliographystyle{theapa}

\end{document}






