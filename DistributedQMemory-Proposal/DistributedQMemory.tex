\documentclass[jair,twoside,11pt,theapa]{article}
\usepackage{jair, theapa, rawfonts}

%\jairheading{1}{2018}{}{}{}
\ShortHeadings{Distributed Q-Memory: A Democratic Memory}
{Robinson}
\firstpageno{1}

\begin{document}

\title{Distributed Q-Memory: A Democratically Assembled Memory (Proposal)}

\author{\name Max Robinson \email max.robinson@jhu.edu \\
       \addr Johns Hopkins University,\\
       Baltimore, MD 21218 USA
   }

% For research notes, remove the comment character in the line below.
% \researchnote

\maketitle

\begin{abstract}
\end{abstract}

\section{Introduction}
\label{Introduction}
Reinforcement learning can be described simply as the learning process of an agent in an environment trying to reach a goal. 
The agent learns by attempting to maximize a reward. The learner has no prior knowledge of the environment and does not know 
which actions to take when. The act of maximizing the reward is the learning process.

Since Reinforcement learning requires a $state \to action \to state'$ loop, the learning process for that agent is inherently sequential. 
As a result in large state spaces or complex games, training can be slow. A single agent must often experience many iterations of maximizing 
a reward in order to learn even a more simple environment.

A famous example of Reinforcement Learning is Tesauro's TD-Gammon agent.
The best performing agent required 1,500,000 training games to beat one of the best Backgammon players at the time \cite{Tesauro:1995:TDL:203330.203343}.
As a more modern example Mnih et al. developed an algorithm to play Atari 2600 video games called DQN (Mnih et al. 2015). To learn to play each game at a human level or higher
50 million frames were required to train an agent for each game and total including all 49 games in the study about 38 days of game experience. 

The constraint of a lone agent acting sequentially can create situations where training an agent to learn a task can take an exorbitant amount of time. 
To combat this, researchers have focused on ways to adapt these reinforcement learning algorithm to run in parallel to decrease the amount of time it takes a single agent to learn. 

% maybe add? 
A lot of research recently has been around speeding up Deep NN for RL such as DQN and others. Quite a few papers have suggested ways of parallelizing both the computation for these methods as well as distributing actors and learners to run in parallel, which send back gradients to be used to update a centralized parameter store. 

I propose the we step back and explore a slightly more simplistic model of distributed RL using Q-learning with a Q-value database to explore the effects of a full combination of states and the multiple affects of distributing learners, such as state exploration rates, effects of learner contributions at different stages of learning, and different models for updating the distributed learners and their effect on exploration and performance. 
%maybe add? ^^^

To add to these works of research I suggest a parallelized and distributed version of the Q-learning algorithm using a traditional Q-Memory, Distributed Q-learning (DQL).
In this distributed form of Q-learning, there is a centralized Q-Memory that is updated by agents that are running in parallel. Each separate agents runs with their own copy of the environment and a Q-memory. Each agent then learns as usual according to the Q-learning algorithm \cite{watkins}. However, every so often an agent will send updates to the centralized 
Q-Memory. 
%These updates are the state's and their q-values that were modified since the last update. 
The centralized Q-memory then is calculates updates to its values, using a linear combination of q-values and hyper parameters explained in Section \ref{Approach}. 

%I hypothesize that in distributing the learning process amongst multiple learners and aggregating their experiences
I hypothesize that in using this distributed algorithm, there will be an increase in the learning rate of a learner using the centralized Q-memory when compared to a single Q-learning agent, as the number of parallel agents increases. This increased learning rate will reduce the total number of epochs per single agent to achieve similar or better performance. This will also likely reducing total wall time it takes to train. 

\section{Previous Work}
\label{Literature Survey}

\section{Approach}
\label{Approach}
The focus of this proposed experiment will be on testing the distributed Q-learning algorithm described in Section \ref{algorithm}.
The new algorithm and its variations will be tested according to the experimental approach described in \ref{experiments}

\subsection{Distributed Q-learning Algorithm} 
\label{algorithm}
The proposed algorithm to study is a novel distributed and parallelized version of Q-learning, referred to as DQL. 
DQL uses multiple agents environment pairs to learn rather than a single agent and environment. A central Q-memory $Q_c$ server is also kept and is updated 
as the agents learns and completes epochs. In addition to the central Q-memory, there is also a central learning rate $\alpha_c$ which decays according
to the number of updates $Q_c$ has received. 

More formally DQL has $n$ agents $A_1, ... , A_n$. Each agent also has a copy of the environment in which it acts, $E_1, ..., E_n$. Each agent keeps track 
of a local Q-Memory $Q_n$, in addition to local parameters such as the discount factor, learning rate, and $\epsilon$. Each agent then acts and learns inside side of
its environment according to the $\epsilon$-greedy Q-learning algorithm. 

As the agent learns it will send updates to $Q_c$. In order to send updates, A list of $Q_i(s,a)$s that have been modified since the agent last sent updates is kept by the local agent,
$Set_{q_i} = Set(Q'_i(s_,a), ... )$.
The agent then sends updates to $Q_c$ every $\tau$ epochs. $\tau$ is a hyperparameter that can be adjusted to change how frequently each agent reports updates. 

Updates are then performed by the $Q_c$ server each time an update is retrieved. $Q_c$ will be updated according to equation \ref{eq1},
where $\alpha_{c}$ and $\alpha_i$ are the learning rate for $Q_c$ and $A_i$ respectively. States in the update set 
\begin{equation}
\label{eq1}
Q_c(s,a) = (1-\frac{\alpha_{c}^{2}}{\alpha_i})Q_c(s,a) + \frac{\alpha_{c}^{2}}{\alpha_i} Q_i(s,a)
\end{equation}

After updates to $Q_c$ are complete, there are two variations that will be explored in this experiment for updating the local agent $A_i$. 
The first variation will send back a set of updated values from $Q_c$ for just the set of states sent from $A_i$, $Set_{q_i}$. The agent will 
then replace any local values with those from $Q_c$ This has the affect of reconciling just the states that local agent has updated recently 
with possible updates from other agents. 
The second variation, dubbed DQL-All, overwrites $Q_i$ with the entire $Q_c$ memory. This provides information from other agents to a local agent and homogenizes 
the $Q_i$s of the different agents. In this variation, states that were explored by one agent would then be shared with other agents even if that 
agent has not explored the states itself. 

Learning halts after $Q_c$ has been updated a maximum number of times, or performance ceases to improve. To measure the performance of $Q_c$ an agent 
and environment $A_t, E_t$ are created with $Q_c$ as the instantiated Q-memory. The agent then acts in the environment with no learning occurring. 
The performance metric being used is than captured upon completion and the test agent and environment are destroyed. How often the performance of $Q_c$ 
is assessed is configurable and depends upon the fidelity with which the trainer wishes to assess the learning. 


\subsection{Experimental Approach}
\label{experiments}
DQL will be evaluated by running experiments on 3 different environments provided by the OpenAI Gym \cite{gym}.
The 3 environments used will be Taxi-v2, LunarLander-v2, and BipedalWalker-v2. The Taxi-v2 environment is a discrete 
environment, while the LunarLander-v2, and BipedalWalker-v2 are continuous. For the continuous state space environments, 
the values for the states will be discretized on a per environment basis to ensure a single learner can learn in the specific environment. 
Each environment supplies a different level of problem complexity, and as such aims to demonstrate that any performance gains that could
be observed using DQL are not subject to the complexity of the environment.

Experiments in each environment will be done using DQL with 2, 4, and 8 agents and done with each variant DQL and DQL-All . Each experiment will be run with 10 trials. The final results
being averaged over all trials. Each agent will use an $\epsilon$-greedy behavior policy with $\epsilon$ annealed from 1 to .1 over the learning process. The annealing may vary per environment. 

The learning rate for $Q_c$ will be annealed from 1 to .1 over the number of updates from agents. The learning rate per agent is annealed from 1 to .1 over the learning process, updating the learning rate after each epoch. The update frequency for each agent, $\tau$, is anticipated to be set to 1 for all experiments. If additional time is available, $\tau$ will be varied 
to use the values 1, 5, and 10 to inspect the effect that update frequency has on learning. 

\subsubsection{Evaluation}
The performance of DQL and DQL-All will be evaluated based on their learning rate to achieve better or equal performance to a single agent system using Q-learning. 
The learning rate will be measured in the number of epochs required to attain the performance. 

The performance measures to evaluate an algorithm in their environment is determined by the provided OpenAI environment. Each environment has a score mechanic that is calculated
differently for each one. A learners performance will be judged on the maximization of the environments scoring metric. For example if the goal is to travel a long distance and the score 
is based on how far the agent got, the performance for that agent will be judge on maximizing the score, i.e. distance. 

In addition to traditional performance metrics, the number of newly explored states will be tracked for each agent. As the agent discovers a new state, the state will be tagged with which epoch
the state was discovered in. This will then be used to compare the exploration patterns of the agents throughout the training process. Comparing which states are discovered by which agents at what time
will give insight into if or when the agents started to converge to the same policies. In addition, it will provide data to evaluate if parallel exploration of the state space provides value to learning. 



\vskip 0.2in
\bibliography{DistributedQMemory}
\bibliographystyle{theapa}

\end{document}






